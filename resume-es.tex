\documentclass[]{mcdowell-cv/mcdowellcv}

\usepackage[english,spanish]{babel}
\usepackage{hyperref}

\name{Luis Ángel\linebreak Prado Postigo}
\address{Perú, Arequipa}
\contacts{(+51) 987 704 882 \linebreak lpradopostigo@gmail.com}

\begin{document}
    \makeheader

    \begin{cvsection}{Sumario}
        \begin{cvsubsection}{}{}{}
            Soy un profesional con más de 3 años de experiencia.
            A lo largo de mi carrera, he mantenido y construido múltiples proyectos, incluidos tiendas en línea, plataformas bancarias y sistemas de gestión.\\\\
            Tengo experiencia trabajando con tecnologías tanto de frontend como de backend, tales como: JavaScript, TypeScript, Go, React, React Native, Node.js, Sass, HTML, CSS, Tailwind, PostgreSQL, SQLite, Docker y Git
        \end{cvsubsection}
    \end{cvsection}

    \begin{cvsection}{Experiencia Profesional}
        \begin{cvsubsection}{Desarrollador Full Stack}{NeoDev}{Febrero 2024 - Presente}
            Kenti: Plataforma B2B que simplifica los pagos entre empresas y sus clientes
            \begin{itemize}
                \item Desarrollador principal del backend, encargado de arreglar bugs e implementar nuevas funcionalidades.
                \item Desarrollé la infraestructura backend para la plataforma que incluye:
                \begin {itemize}
                    \item Diseño de esquema de base de datos y gestión de migraciones
                    \item Diseño, implementación y documentación de API
                    \item Integraciones de servicios de terceros tales como Zipago, Cloudflare y Resend
                \end{itemize}
            \end{itemize}
        \end{cvsubsection}

        \begin{cvsubsection}{Desarrollador Frontend}{AccountTECH}{Mayo 2022 - Enero 2024}
            Darwin Cloud: Plataforma B2B para ayudar a las empresas de inmobiliarias a gestionar sus negocios
            \begin{itemize}
                \item Desarrollé la librería de componentes de la plataforma, brindando consistencia a la experiencia de usuario.
                \item Arreglé errores críticos dentro del sistema de compilación, estandarice los procesos de manejo de formularios y corregí las inconsistencias de estilo.
                \item Escribí documentación para el desarrollo de nuevas funcionalidades.
                \item Corregí bugs e implementé nuevas funcionalidades tales como: documentos de usuario, autenticación de dos pasos, pagos de agentes, etc.
            \end{itemize}
        \end{cvsubsection}

        \begin{cvsubsection}{Analista Programador}{Bantotal}{Mayo 2021 - Abril 2022}
            Efectiva: Aplicación de banca móvil
            \begin{itemize}
                \item Desarrollador principal, encargado de arreglar bugs e implementar nuevas funcionalidades.
                \item Actualicé la apariencia de la aplicación.
                \item Implementé las funcionalidades de onboarding, historial de transacciones y autenticación de dos pasos.
            \end{itemize}

            Alfin: Aplicación de banca móvil
            \begin{itemize}
                \item Corregí bugs e implementé las funcionalidades de onboarding, transferencia de fondos, campañas y autenticación de dos pasos.
            \end{itemize}

        \end{cvsubsection}
    \end{cvsection}

    \begin{cvsection}{Educación}
        \begin{cvsubsection}
        {Perú, Arequipa}
        {\mbox{Universidad Nacional de San Agustín}}
        {Abril 2019 - Abril 2022}
            Ciencia de la Computación
        \end{cvsubsection}
    \end{cvsection}

    \begin{cvsection}{Idiomas}
        \begin{cvsubsection}{}{}{}
            \begin{itemize}
                \item Español (Nativo)
                \item Inglés (C1)
            \end{itemize}
        \end{cvsubsection}
    \end{cvsection}

    \begin{cvsection}{Enlaces}
        \begin{cvsubsection}{}{}{}
            \begin{itemize}
                \item \href{https://www.linkedin.com/in/luis-angel-prado-postigo-813916231/}{Linkedin}
                \item \href{https://luis-prado-portfolio.netlify.app/}{Portafolio}
                \item \href{https://github.com/lpradopostigo}{Github}
            \end{itemize}
        \end{cvsubsection}
    \end{cvsection}
\end{document}
