\documentclass[]{mcdowell-cv/mcdowellcv}

\usepackage[english,spanish]{babel}
\usepackage{hyperref}

\directlua{dofile("macro.lua")}

\name{Luis Ángel \linebreak Prado Postigo}
\address{\es{Perú}\en{Peru}, Arequipa}
\contacts{(+51) 987-704-882 \linebreak lpradopostigo@gmail.com)}

\newcommand{\proficient}{(\en{proficient}\es{proficiente})}
\newcommand{\experience}{(\en{prior experience}\es{experiencia previa})}

\begin{document}
    \makeheader

    \begin{cvsection}{\en{Employment}\es{Carrera}}
        \begin{cvsubsection}
        {\en{Front End Developer}\es{Desarrollador Front End}}
        {Bantotal}
        {1 \en{year}\es{año}}
            Alfin
            \begin{itemize}
                \item
                \en{Main developer in the development and production phase}
                \es{Desarrollador principal en la fase de desarrollo y producción}

                \item
                \en{Onboarding and password recovery functionality implementation}
                \es{Implementación de las funcionalidades de onboarding y recuperación de contraseña}
            \end{itemize}

            Efectiva
            \begin{itemize}
                \item
                \en{Main developer in the development phase}
                \es{Desarrollador principal en la fase de desarrollo}

                \item
                \en{Onboarding implementation}
                \es{Implementación de la funcionalidad de onboarding}

                \item
                \en{New design implementation}
                \es{Implementación del nuevo diseño}
            \end{itemize}

        \end{cvsubsection}
    \end{cvsection}

    \begin{cvsection}{\en{Education}\es{Educación}}
        \begin{cvsubsection}
        {Arequipa, \es{Perú}\en{Peru}}
        {\mbox{Universidad Nacional de San Agustín}}
        {2019 - \en{present}\es{presente}}
            \en{Computer Science}
            \es{Ciencia de la Computación}
        \end{cvsubsection}
    \end{cvsection}

    \begin{cvsection}{\en{Technical Experience}\es{Experiencia Técnica}}
        \begin{cvsubsection}{\en{Projects}\es{Proyectos}}{}{}
            \begin{itemize}
                \item \textbf{\en{Desktop music player}\es{Reproductor de música para escritorio}}
                \en{Has gapless playback and music library functionality}
                \es{Reproduce música sin pausas entre canción y canción, posee funcionalidad de librería}
            \end{itemize}

            \begin{itemize}
                \item \textbf{\en{Audio playback library}\es{Librería para reproducir audio}}
                \en{Provides an easy way to achieve gapless playback in Node.js, supports most popular audio formats (mp3, flac, m4a, etc...)}
                \es{Provee una forma fácil de reproducir audio sin pausas en Node.js, soporta los formatos más populares (mp3, flac, m4a, etc...)}
            \end{itemize}

            \begin{itemize}
                \item \textbf{\en{Online image encryption tool}\es{Herramienta para encriptar imágenes online}}
                \en{Uses a symmetric encryption algorithm that runs on the client side}
                \es{Usa un algoritmo de cifrado simétrico que es ejecutado en lado del cliente}
            \end{itemize}

            \begin{itemize}
                \item \textbf{\en{Online music player}\es{Reproductor de música online}}
                \en{Has audio tags visualization functionality}
                \es{Posee funcionalidad para visualización de tags de audio}
            \end{itemize}
        \end{cvsubsection}
    \end{cvsection}

    \begin{cvsection}{\en{Languages and Technologies}\es{Lengüajes y Tecnologías}}
        \begin{cvsubsection}{}{}{}
            \begin{itemize}
                \item JavaScript \proficient, HTML \proficient, CSS \proficient, C++ \proficient, Sass \experience, Typescript \experience, SQLite \experience, python \experience
                \item React \proficient, React Native \proficient, Node.js \proficient, Electron \proficient, Git \proficient, Tailwind \proficient, Figma \experience, AdobeXD \experience
            \end{itemize}
        \end{cvsubsection}
    \end{cvsection}

    \begin{cvsection}{\en{Links}\es{Enlaces}}
        \begin{cvsubsection}{}{}{}
            \begin{itemize}
                \item \href{https://lpradopostigo-portfolio.netlify.app/
}{\en{Portfolio}\es{Portafolio}}
                \item \href{https://www.linkedin.com/in/luis-angel-prado-postigo-813916231/
}{Linkedin}
                \item \href{https://github.com/lpradopostigo
}{Github}
            \end{itemize}
        \end{cvsubsection}
    \end{cvsection}
\end{document}
